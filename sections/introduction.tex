\begin{mdframed}[innertopmargin=10pt, innerbottommargin=10pt, innerleftmargin=10pt, innerrightmargin=10pt, skipabove=10pt, skipbelow=10pt, roundcorner=10pt]
    \textbf{Introduktion}
    
    Detta kapitel ska ge läsaren en introduktion till arbetet. Vanliga underrubriker (men alla arbeten behöver nödvändigtvis inte innehålla alla underrubriker eller i denna ordning):
    
    \begin{itemize}
        \item Bakgrund (motivering till arbetet och frågeställningen)
        \item Problemformulering
        \item Syfte
        \item Frågeställning 
        \item Avgränsning
        \item Kravspecifikation (ev. vid vissa arbeten)
        \item Ordlista
    \end{itemize}
    \end{mdframed}

\section{Background}
\label{sec:background}
\ac{ros} \cite{Eros2019}

\section{Problem Formulation}
\label{sec:problemformulation}
\input{sections/problemformulation.tex}

\section{Purpose}
\label{sec:purpose}
%\input{sections/purpose.tex}

\section{Research Objectives}
\label{sec:researchobjectives}
\input{sections/researchobjectives.tex}

\section{Limitations}
\label{sec:limitations}
\input{sections/limitations.tex}

